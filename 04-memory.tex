\section{存储器层次结构}
\subsection{存储技术}

\begin{sidenote}{局部性原理}
    局部性原理(Locality Principle)是计算机科学中的一个重要概念,指的是在程序执行过程中,访问的数据和指令往往集中在某些特定的区域。这种现象可以分为两种类型:时间局部性和空间局部性。
    
    \textbf{时间局部性}(Temporal Locality)指的是如果一个数据项在某个时间点被访问,那么在不久的将来它很可能会再次被访问。换句话说,最近使用过的数据很可能会再次被使用。
    
    \textbf{空间局部性}(Spatial Locality)指的是如果一个数据项在某个时间点被访问,那么与它相邻的数据项也很可能会在不久的将来被访问。换句话说,程序倾向于访问存储器中相近的地址。
    
    局部性原理是设计高效存储器层次结构(如缓存、主存和辅助存储器)的基础。通过利用局部性原理,计算机系统可以显著提高数据访问速度,减少延迟,从而提升整体性能。

\end{sidenote}
\subsection{缓存}
