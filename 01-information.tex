\section{信息的表示和处理}

\subsection{信息存储}

\paragraph{进制转换}
\begin{table}[htbp]
    \centering
    \begin{tabular}{|c|c|c|}
        \hline
        十六进制 & 十进制 & 二进制  \\
        \hline
        0x0  & 0   & 0000 \\
        \hline
        0x1  & 1   & 0001 \\
        \hline
        0x2  & 2   & 0010 \\
        \hline
        0x3  & 3   & 0011 \\
        \hline
        0x4  & 4   & 0100 \\
        \hline
        0x5  & 5   & 0101 \\
        \hline
        0x6  & 6   & 0110 \\
        \hline
        0x7  & 7   & 0111 \\
        \hline
        0x8  & 8   & 1000 \\
        \hline
        0x9  & 9   & 1001 \\
        \hline
        0xA  & 10  & 1010 \\
        \hline
        0xB  & 11  & 1011 \\
        \hline
        0xC  & 12  & 1100 \\
        \hline
        0xD  & 13  & 1101 \\
        \hline
        0xE  & 14  & 1110 \\
        \hline
        0xF  & 15  & 1111 \\
        \hline
    \end{tabular}
\end{table}

\paragraph{基本数据类型的存储}
\begin{table}[htbp]
    \centering
    \begin{tabular}{|c|c|c|}
        \hline
        数据类型    & 32位 & 64位 \\
        \hline
        char    & 1   & 1   \\
        \hline
        short   & 2   & 2   \\
        \hline
        int     & 4   & 4   \\
        \hline
        long    & 4   & 8   \\
        \hline
        float   & 4   & 4   \\
        \hline
        double  & 8   & 8   \\
        \hline
        pointer & 4   & 8   \\
        \hline
    \end{tabular}
\end{table}
C语言中\emph{字符串}被编码为一个以null(其值为0)字符结尾的字符数组。每个字符都由ASCII字符码来表示。
注意,十进制数字x的ASCII码正好是0x3x, 而终止字节的十六进制表示为0x00。

\paragraph{寻址和字节顺序}
多字节对象会被存储为连续的字节序列,对象的地址为所使用字节中最小的地址。按照从最低
有效字节到最高有效字节的顺序存储对象的方法称为\emph{小端法},反之则称为\emph{大端法}。
\begin{itemize}
    \item 小端法:x86,ARM等。
    \item 大端法:Sun,Internet,PPC Mac等。
\end{itemize}

\subsection{信息处理}

\paragraph{位级运算}
\begin{itemize}
    \item 按位与(AND)
    \item 按位或(OR)
    \item 按位异或(XOR)
    \item 按位取反(NOT)
\end{itemize}

\begin{sidenote}{位级运算的完备性}
    位级运算中的按位与(AND)、按位或(OR)和按位取反(NOT)是完备的,即可以通过它们的组合实现任何其他位级运算。
    例如,按位异或(XOR)可以通过以下表达式实现:
    $$ x \oplus y = (x \& \sim y) | (\sim x \& y) $$
    常见的完备集有:
    \begin{itemize}
        \item $\{AND, NOT\}$
        \item $\{OR, NOT\}$
        \item $\{NAND\}$
        \item $\{NOR\}$
    \end{itemize}
\end{sidenote}

\paragraph{逻辑运算}
\begin{itemize}
    \item 逻辑与(AND)
    \item 逻辑或(OR)
    \item 逻辑非(NOT)
\end{itemize}

短路机制:在逻辑与(AND)和逻辑或(OR)运算中,如果第一个操作数已经能够确定整个表达式的值,则第二个操作数将不会被计算。

\paragraph{移位运算}
\begin{itemize}
    \item 左移
    \item 逻辑右移
    \item 算术右移
\end{itemize}

\begin{sidenote}{位运算的优先级}
    位运算符的优先级顺序从高到低依次为:
    \begin{enumerate}
        \item 按位取反(\verb|~|)
        \item 加减乘除模(\verb|+ - * / %|)
        \item 左移(\verb|<<|)和右移(\verb|>>|)
        \item 按位与(\verb|&|)
        \item 按位异或(\verb|^|)
        \item 按位或(\verb!|!)
    \end{enumerate}
    细节可参看\href{https://c-cpp.com/c/language/operator_precedence}{C-CPP}。
\end{sidenote}

\subsection{整数}
\subsubsection{整数表示}
\paragraph{类型}
\begin{itemize}
    \item 无符号整数(unsigned int)
          $$x = \sum_{i=0}^{w-1} b_i 2^i$$
          $$UMin = 0, UMax = 2^w - 1$$
    \item 有符号整数(signed int)
          $$x = -b_{w-1} 2^{w-1} + \sum_{i=0}^{w-2} b_i 2^i$$
          $$TMin = -2^{w-1}, TMax = 2^{w-1} - 1$$
\end{itemize}

处理同样字长的有符号数和无符号数之间相互转换的一般规则是:数值可能会改变,但是位模式不变。

\paragraph{拓展}
\begin{itemize}
    \item 零拓展(zero extension):用于无符号数的拓展,高位补 0 。
    \item 符号拓展(sign extension):用于有符号数的拓展,高位补符号位。
\end{itemize}
\paragraph{截断}
截断是指将一个较长字长的整数转换为较短字长的整数。截断时,只保留低位的若干位,高位被丢弃。

\begin{sidenote}{C语言中的隐式类型转换}
    在C语言中,不同类型整数进行比较时,所有比 int 类型小的整数类型(如 char、short 等)会首先被提升为 int 类型。
    如果两个数宽度不同(例如 int 和 long),宽度较小的类型会转换为宽度较大的类型。
    最后,如果比较涉及有符号类型和无符号类型,有符号类型会被转换为无符号类型再进行对比。
\end{sidenote}

\subsubsection{整数运算}
\paragraph{加减法}

无符号数:
$$
    x +_w^u y =
    \begin{cases}
        x + y       & x + y < 2^w    \\
        x + y - 2^w & x + y \geq 2^w
    \end{cases}
$$
$$
    -_w^u x =
    \begin{cases}
        x       & x = 0 \\
        2^w - x & x > 0
    \end{cases}
$$

符号数:
$$
    x +_w^t y =
    \begin{cases}
        x + y - 2^w & 2^{w-1} \leq x + y         \\
        x + y       & -2^{w-1} < x + y < 2^{w-1} \\
        x + y + 2^w & x + y \leq -2^{w-1}
    \end{cases}
$$
$$
    -_w^t x =
    \begin{cases}
        TMin_w & x = TMin_w \\
        - x    & x > Tmin_w
    \end{cases}
$$
\paragraph{乘除法}

无符号数:
\begin{itemize}
    \item 乘法
          $$
              x *_w^u y = (x \cdot y) mod 2^w
          $$
    \item 乘2的幂
          $$
              x *_w^u 2^k = x << k
          $$
    \item 除以2的幂(逻辑右移,向下舍入)
          $$
              x /_w^u 2^k = x >> k
          $$
\end{itemize}




符号数:
\begin{itemize}
    \item 乘法
          $$
              x *_w^t y = U2T_w((x \cdot y) mod 2^w)
          $$
    \item 乘2的幂
          $$
              x *_w^t 2^k = x << k
          $$
    \item 除以2的幂(算术右移,向下舍入)
          $$
              x /_w^t 2^k = x >> k
          $$
    \item 除以2的幂(算术右移,向上舍入)
          $$
              x /_w^t 2^k = (x + (1 << k) - 1) >> k
          $$
\end{itemize}

注意,C语言中的$/$运算符对于整数除法采用向零舍入方式。

\begin{sidenote}{整数乘法的二进制低位结果}
    假设有两个 $n$ 位的二进制数,其位模式分别为 $A$ 和 $B$。作为无符号整数,它们的值即为 $A$ 和 $B$。
    它们的数学乘积是 $A \cdot B$。在计算机中,仅保留低 $n$ 位,相当于计算 $(A \cdot B) \mod 2^n$。
    若将 $A$ 和 $B$ 解释为有符号整数(补码),设 $a = A - 2^n \cdot \text{sign}(A)$,$b = B - 2^n \cdot \text{sign}(B)$,其中 $\text{sign}(X)$ 在 $X$ 为负数时为 1,否则为 0。
    有符号乘积的低 $n$ 位为 $(a \cdot b) \mod 2^n$,与无符号乘积的低 $n$ 位相同。推导如下:
    \begin{align*}
        (a \cdot b) \mod 2^n & = [(A - 2^n \cdot \text{sign}(A)) \cdot (B - 2^n \cdot \text{sign}(B))] \mod 2^n \\
                             & = (A \cdot B) \mod 2^n
    \end{align*}
\end{sidenote}

\subsection{浮点数}
\subsubsection{浮点数表示}

\paragraph{二进制小数}
$$ b = \sum_{k=-j}^{i} b_k \cdot 2^k $$

\paragraph{IEEE 754标准}
$$V = (-1)^{s} \cdot M \cdot 2^E$$
\begin{itemize}
    \item 符号(sign):s 表示数值的正负。
    \item 阶码(exponent):E 的作用是对浮点数加权,这个权重是 2 的 E 次幂。
    \item 尾数(mantissa):M 是一个二进制小数。
\end{itemize}
将浮点数的位表示划分为三个字段,分别对这些值进行编码:
\begin{itemize}
    \item 一个单独的符号位 s 直接编码符号$s$ 。
    \item k 位的阶码字段 $exp=e_{k-1}\ldots e_1e_0$ 编码阶码 E 。
    \item n 位小数字段 $frac=f_{n-1}\ldots f_1f_0$ 编码尾数 M ,但是编码出来的值也依赖于阶码字段的值是否等于 0 。
\end{itemize}
\begin{table}[htbp]
    \centering
    \begin{tabular}{|c|c|c|c|c|}
        \hline
        类型          & 总位数 & 符号位(s) & 阶码位(k) & 尾数位(n) \\
        \hline
        单精度(float)  & 32  & 1      & 8      & 23     \\
        \hline
        双精度(double) & 64  & 1      & 11     & 52     \\
        \hline
    \end{tabular}
\end{table}

\paragraph{分类}
\begin{itemize}
    \item 规格化的值(Normalized):阶码字段不全为 0 ,尾数字段有有效数字。
          $$E=exp - Bias$$
          $$M=1.f_{n-1}f_{n-2}\ldots f_0$$
          其中,$Bias = 2^{k-1} - 1$。
    \item 非规格化的值(Denormalized):阶码字段全为 0 ,尾数字段有有效数字。
          $$E = 1 - Bias = - (2^{k-1} - 2)$$
          $$M = 0.f_{n-1}f_{n-2}\ldots f_0$$
    \item 无穷大(Infinity):阶码字段全为 1 ,尾数字段全为 0 。
    \item NaN(Not a Number):阶码字段全为 1 ,尾数字段非全 0 。
\end{itemize}

\subsubsection{浮点数运算}
\paragraph{舍入}
\begin{itemize}
    \item 向零舍入
    \item 向上舍入
    \item 向下舍入
    \item 向偶数舍入(默认)
\end{itemize}

\paragraph{乘法}
$$ (-1)^{s_1}M_1 2^{E_1} \times (-1)^{s_2}M_2 2^{E_2} $$
\begin{enumerate}
    \item 符号计算:结果的符号位为两个操作数符号位的异或。
          $$s = s_1 \oplus s_2$$
    \item 阶码相加:将两个操作数的阶码相加。
          $$E = E_1 + E_2$$
    \item 尾数相乘:将两个操作数的尾数相乘。
          $$M = M_1 \times M_2$$
    \item 规格化:如果结果的尾数不在规范化范围内,则对其进行规格化处理。
    \item 溢出检查:如果结果的阶码超出表示范围,则将结果设为无穷大。
    \item 舍入:根据所选的舍入模式对结果进行舍入。
\end{enumerate}

\paragraph{加法}
$$ (-1)^{s_1}M_1 2^{E_1} + (-1)^{s_2}M_2 2^{E_2}, E_1 \geq E_2$$
\begin{enumerate}
    \item 对阶:将阶码较小的数的尾数右移,直到两个数的阶码相等。
    \item 尾数相加:根据符号位对尾数进行加减运算。
          $$M = (-1)^{s_1} M_1 + (-1)^{s_2} M_2 2^{E_2 - E_1}$$
    \item 规格化:如果结果的尾数不在规范化范围内,则对其进行规格化处理。
    \item 溢出检查:如果结果的阶码超出表示范围,则将结果设为无穷大。
    \item 舍入:根据所选的舍入模式对结果进行舍入。
\end{enumerate}

\subsubsection{浮点数转换}
\begin{itemize}
    \item int - float:数字不会溢出,但是可能被舍入。
    \item int/float - double:数字不会溢出,也不会舍入。
    \item double - float:可能会溢出,也可能会舍入。
    \item float/double - int:会向零舍入,可能会溢出。
\end{itemize}

\newpage