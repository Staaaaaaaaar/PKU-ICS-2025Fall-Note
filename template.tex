% 引入必要的包
% \usepackage[utf8]{inputenc}
\usepackage{amsmath, amssymb, amsfonts,amsthm}
\usepackage{graphicx}
\usepackage{geometry}
\usepackage{fancyhdr}
\usepackage{titlesec}
\usepackage{hyperref}
\usepackage{xcolor}
\usepackage{listings}
\usepackage{booktabs}
\usepackage{caption}
\usepackage{enumitem}
\usepackage{tcolorbox}
\usepackage{eso-pic}
\usepackage{fontawesome5}
\usepackage{tocloft}
\usepackage{multirow}
\usepackage{float}
\usepackage{subcaption}

% 文献引用
\usepackage[backend=biber,style=numeric,citestyle=numeric]{biblatex}
\addbibresource{references.bib}  % 引入 references.bib 文件

% 设置图片目录
\graphicspath{{Figures/}}

% 页面布局设置
\geometry{left=2.5cm, right=2.5cm, top=2.5cm, bottom=2.5cm}
\setlength{\headheight}{14.5pt}
\addtolength{\topmargin}{-2.5pt}

% 添加封面图片
% \AddToShipoutPicture*{
%     \put(0,0){%
%         \includegraphics[width=\paperwidth,height=\paperheight]{Figures/cover.jpg}}
% }

% 自定义标题样式
\definecolor{sectioncolor}{RGB}{44, 102, 203}
\definecolor{subsectioncolor}{RGB}{0, 128, 128}
\definecolor{subsubsectioncolor}{RGB}{70, 130, 180}
\definecolor{paragraphcolor}{RGB}{105, 105, 105}

% 一级标题
\titleformat{\section}
{\normalfont\Large\bfseries\color{sectioncolor}}
{\thesection}{1em}{}
% 二级标题
\titleformat{\subsection}
{\normalfont\large\bfseries\color{subsectioncolor}}
{\thesubsection}{1em}{}
% 三级标题
\titleformat{\subsubsection}
{\normalfont\normalsize\bfseries\color{subsubsectioncolor}}
{\thesubsubsection}{1em}{}
% 四级标题
\titleformat{\paragraph}
{\normalfont\normalsize\bfseries\color{paragraphcolor}}
{\theparagraph}{1em}{}

% 修改目录的字体颜色为黑色
\hypersetup{
  colorlinks=true,
  linkcolor=black,  % 将目录超链接设置为黑色
  urlcolor=magenta,
  citecolor=black,
  pdfborder={0 0 0}
}

% 页眉页脚设置
\pagestyle{fancy}
\fancyhf{}
\fancyhead[L]{\leftmark}
\fancyhead[R]{\thepage}

% 重定义 plain 样式(用于统一目录页)
\fancypagestyle{plain}{
  \fancyhf{}
  \fancyhead[L]{\leftmark}
  \fancyhead[R]{\thepage}
}

% 代码设置
\definecolor{codebg}{RGB}{248,248,248}        % 代码背景
\definecolor{codeframe}{RGB}{220,220,220}     % 边框颜色
\definecolor{codekeyword}{RGB}{0,0,180}       % 关键字
\definecolor{codecomment}{RGB}{100,120,100}   % 注释
\definecolor{codestring}{RGB}{163,21,21}      % 字符串
\definecolor{codeemph}{RGB}{128,0,128}        % 强调
\lstset{
  % 基础字体与布局
  basicstyle=\ttfamily\small,               % 字体
  backgroundcolor=\color{codebg},           % 背景色
  frame=single,
  rulecolor=\color{codeframe},
  columns=fullflexible,
  keepspaces=true,
  showstringspaces=false,
  tabsize=4,
  breaklines=true,
  breakatwhitespace=true,
  % 行号
  numbers=left,
  numberstyle=\scriptsize\color{gray},
  numbersep=8pt,
  % 配色
  keywordstyle=\bfseries\color{codekeyword},
  commentstyle=\color{codecomment},
  stringstyle=\color{codestring},
  % 额外
  captionpos=b,
  escapeinside={(*@}{@*)},                  % 在代码中嵌入 LaTeX:(*@  LaTeX  @*)
  inputencoding=utf8,
  extendedchars=true
}
% C语言风格
\lstdefinestyle{CStyle}{
  language=C,
  morekeywords={inline,__inline,__attribute__,constexpr,decltype,nullptr,
      size_t,uint8_t,uint16_t,uint32_t,uint64_t,int8_t,int16_t,int32_t,int64_t},
  % 将 # 也视为行注释的开始符号
  morecomment=[l][\color{codecomment}]{\#}
}
% 汇编语言风格
\lstdefinestyle{ASMStyle}{
  language={[x86masm]Assembler},
  keywordstyle=\bfseries\color{purple},
  emph={eax,ebx,ecx,edx,esi,edi,esp,ebp,rax,rbx,rcx,rdx,rsi,rdi,rsp,rbp,r8,r9,r10,r11,r12,r13,r14,r15},
  emphstyle=\color{codeemph},
  morekeywords={
      mov,movb,movw,movl,movq,movsb,movsw,movsl,movsq,movzx,movsx,movabs,
      add,addb,addw,addl,addq,adc,adcb,adcw,adcl,adcq,
      sub,subb,subw,subl,subq,sbb,sbbb,sbbw,sbbl,sbbq,
      imul,imulb,imulw,imull,imulq,
      mul,mulb,mulw,mull,mulq,div,divb,divw,divl,divq,idiv,idivb,idivw,idivl,idivq,
      inc,incb,incw,incl,incq,dec,decb,decw,decl,decq,
      xor,xorb,xorw,xorl,xorq,and,andb,andw,andl,andq,or,orb,orw,orl,orq,not,neg,
      shl,shlb,shlw,shll,shlq,sal,salb,salw,sall,salq,
      shr,shrb,shrw,shrl,shrq,sar,sarb,sarw,sarl,sarq,
      rol,rolb,rolw,roll,rolq,ror,rorb,rorw,rorl,rorq,
      lea,leaq,leaw,leal,leab,
      push,pushb,pushw,pushl,pushq,pop,popb,popw,popl,popq,
      call,ret,retq,int,int3,iret,iretq,syscall,sysenter,sysexit,sysret,sysretq,
      cmp,cmpb,cmpw,cmpl,cmpq,test,testb,testw,testl,testq,
      jmp,je,jne,jz,jnz,ja,jae,jb,jbe,jl,jle,jg,jge,jnc,jc,jo,jno,js,jns,jp,jnp,
      nop,hlt,
      section,global,extern,org,align,db,dw,dd,dq,byte,word,dword,qword,
      pushf,popf,pusha,popa,cld,std,stc,clc,sti,cli,rep,repz,repe,repne,repnz,
      cmova,cmovb,cmovc,cmovg,cmovl,cmovge,cmovle,cmovbe,cmovnb,cmovns,cmovnz,cmovo,
      setb,setbe,setl,setle,setg,setge,setnz,setz,sets,setns,seto,
      bt,bts,btr,btc,bsf,bsr,bswap,
      pxor,pand,por,psllw,psrlw,pslld,psrld,pslldq,psrldq,psadbw,
      fld,fstp,fst,fstcw,fninit,fwait,
      emms,mmx,fxsave,fxrstor
    }
}

% 设置自定义样式的定理框
\tcbuselibrary{theorems}
% 定义定理的样式
\newtcolorbox{dl}[1][]{colback=red!5!white,colframe=red!75!black,
  fonttitle=\bfseries,title=定理, #1}

% 定义定义的样式
\newtcolorbox{dy}[1][]{colback=green!5!white,colframe=green!75!black,
  fonttitle=\bfseries,title=定义, #1}

% 定义公式的样式
\newtcolorbox{gs}[1][]{colback=blue!5!white,colframe=blue!75!black,
  fonttitle=\bfseries,title=公式, #1}

% 定义旁注的样式,支持传入参数作为标题
\newtcolorbox{sidenote}[2][]{title=#2, #1}


